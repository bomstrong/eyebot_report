\chapter{Conclusion}
\label{ch:conclusion}

In the previous sections we explored the different blocks and technologies used in this project. We looked at how to capture and process images to detect a line, and how to calculate an error signal from the segmented image. By using a camera, we archived a much more precise and accurate error signal, compared to a regular array of photo sensors. By that we archived a more smoother line following, with performance comparable to real world usage in autonomous vehicles.

%Motors
We have gained a huge insight in the control of DC-motors, both electronics and control theory using feedback loops from encoders. The work has also showed that motors linear on the paper are highly non-linear in the real world.

%Controllers
By implementing discrete PID controllers in the Raspberry Pi, we archived an almost perfect line following, that in all cases fully satisfies the requirement specification. All controllers were tuned by a mixture of Ziegler-Nichols, and the principles mentioned in the section about manual tuning. 

%Track
Completing the track was a question of implementing a state machine, and has actually been one of the easiest when all the individual features of the robot were available, i.e. following the line, turning to a given position, breaking, following the wall and so on.

%Project
In this project we have gained a lot of new useful knowledge in various areas, especially the area of professional PCB production, image processing and control theory. This has not been cost-free, and a huge amount of hours has spent in labs and in front of the screen.

%Further work / improvements
Future work should focus on improving the segmentation between the line and the floor, and possibly implement some more advanced image processing based on edge detection and BLOB feature classification. A more sophisticated detection of the different features on the track, could be implemented to improve and possibly simplify the state machine. 